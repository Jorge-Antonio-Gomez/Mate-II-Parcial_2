\documentclass{article}
\usepackage[utf8]{inputenc}
\usepackage{mathrsfs, amsmath}
    % mathrsfs = símbolos: lagrangiano
        % \mathscr{L}
    % amsmath = romper ecuaciones en nuevas líneas y centrar
        % \begin{split}
        % \end{split}

\title{Producto punto y normas (para vectores, matrices, polinomios y funciones continuas)}
\author{
    Jorge Antonio Gómez García \\
    Emiliano Martín Lugo López \\
    Saud Antonio Morales González
}
\date{Matemáticas II\\03 de abril de 2022}

\begin{document}
\maketitle

    \section{Espacios vectoriales}

         ......................................................................................................

    \section{Definición del producto punto}

        Sea $u$, $v$ y $w$ vectores en un \textbf{espacio vectorial} y sea c cualquier escalar. Un producto interno en $V$ es una función que asocia un número real $\langle u, v\rangle$ con cada par de vectores $u$ y $v$ que cumplen los siguientes:\footnote{Ron Larson, \textit{Algebra lineal. matemáticas} 4, 7a ed. (Ciudad de México: Cengage learning, 2018), 185.}

        \subsection{Axiomas o propiedades del producto punto}

        \begin{enumerate}

            \item $\langle u, v\rangle = \langle v, u\rangle$
            \item $\langle u, v + w\rangle = \langle u, v\rangle + \langle u, w\rangle$
            \item $c\langle u, v\rangle=\langle c u, v\rangle=\langle u, c v\rangle$
            \item $\langle v, v\rangle \geq 0$ y $\langle v, v\rangle = 0$ si y sólo si $v = 0$

        \end{enumerate}

        Cuando un espacio vectorial $V$ tiene al menos un producto punto, podemos llamarlo \textbf{espacio con producto punto}. Ahora bien, cuando se hace referencia a un espacio con producto punto, se supone que \textbf{el conjunto de escalares es el conjunto de los números reales}.

        %

    \section{Producto punto y vectores}
    
        \subsection{Módulo de un vector en términos del producto punto}



        \subsection{Ángulo entre dos vectores en términos de su producto punto}



        \subsection{Ortogonalidad}



    \section{Producto punto y matrices}



    \section{Producto punto y polinomios}



    \section{Producto punto y funciones continuas}



    \section{Ejercicios}

        \subsection{Vectores}



        \subsection{Matrices}



        \subsection{Polinomios}

            Con los siguientes polinomios:
            \begin{gather*}
                p(x)=1-2x^{2}\hspace{35pt}q(x)=4-2x+x^{2}\hspace{35pt}r(x)=x+2x^{2}
            \end{gather*}

            Determine:
            \begin{gather*}
                \textbf{a.}\hspace{2pt}\langle p,q\rangle\hspace{30pt}\textbf{b.}\hspace{2pt}\langle q,r\rangle\hspace{30pt}\textbf{c.}\hspace{2pt}\parallel q\parallel\hspace{35pt}\textbf{d.}\hspace{2pt}d(p,q)
            \end{gather*}

            \textbf{a.}
                \begin{align*}
                    \langle p,q\rangle&= (4)(1)+(0)(-2)+(-2)(1) \\
                    \langle p,q\rangle&= 4+0-2 \\
                    \langle p,q\rangle&= 2 
                \end{align*}

            \textbf{b.}
                \begin{align*}
                    \langle q,r\rangle&= (4)(0)+(-2)(1)+(1)(2)\\
                    \langle q,r\rangle&= 0-2+2\\
                    \langle q,r\rangle&= 0
                \end{align*}

            \textbf{c.}
                \begin{align*}
                    \parallel p\parallel&= \sqrt{\langle q,q\rangle} \\
                    \parallel p\parallel&= \sqrt{(4)(4)+(-2)(-2)+(1)(1)}=\sqrt{4^{2}+(-2)^{2}+1^{2}} \\
                    \parallel p\parallel&= \sqrt{16+4+1} \\
                    \parallel p\parallel&= \sqrt{21}
                \end{align*}

            \textbf{d.}
                \begin{align*}
                    d(p,q)&= \parallel p-q\parallel \\
                    d(p,q)&= \parallel (1-2x^{2})-(4-2x+x^{2})\parallel \\
                    d(p,q)&= \parallel -3+2x-3x^{2}\parallel \\
                    d(p,q)&= \sqrt{(-3)^{2}+2^{2}+(-3)^{2}} \\
                    d(p,q)&= \sqrt{22}
                \end{align*}

        \subsection{Funiones continuas}



    \section{Tarea}

    

\end{document}